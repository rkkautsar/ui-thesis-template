%-----------------------------------------------------------------------------%
\chapter{\babLima}
%-----------------------------------------------------------------------------%
\todo{Tambahkan kata-kata pengantar bab 5 disini.}


%-----------------------------------------------------------------------------%
\section{Mengubah Tampilan Teks}
%-----------------------------------------------------------------------------%
Beberapa perintah yang dapat digunakan untuk mengubah tampilan adalah: 
\begin{itemize}
	\item \bslash f \\
		Merupakan alias untuk perintah \bslash textit, contoh 
		\f{contoh hasil tulisan}.
	\item \bslash bi \\
		\bi{Contoh hasil tulisan}.
	\item \bslash bo \\
		\bo{Contoh hasil tulisan}.
	\item \bslash m \\
		Contoh\ hasil\ tulisan: $\alpha \not= \m{\alpha}$
	\item \bslash code \\ 
		\code{Contoh hasil tulisan}.
\end{itemize}


%-----------------------------------------------------------------------------%
\section{Memberikan Catatan}
%-----------------------------------------------------------------------------%
Ada dua perintah untuk memberikan catatan penulisan dalam dokumen yang Anda 
kerjakan, yaitu: 
\begin{itemize}
	\item \bslash todo \\
		Contoh: \\ \todo{Contoh bentuk todo.}
	\item \bslash todoCite \\ 
		Contoh: \todoCite
\end{itemize}


%-----------------------------------------------------------------------------%
\section{Menambah Isi Daftar Isi}
%-----------------------------------------------------------------------------%
Terkadang ada kebutuhan untuk memasukan kata-kata tertentu kedalam Daftar Isi. 
Perintah \bslash addChapter dapat digunakan untuk judul bab dalam Daftar isi. 
Contohnya dapat dilihat pada berkas thesis.tex.


%-----------------------------------------------------------------------------%
\section{Memasukan PDF}
%-----------------------------------------------------------------------------%
Untuk memasukan PDF dapat menggunakan perintah \bslash inpdf yang menerima satu 
buah argumen. Argumen ini berisi nama berkas yang akan digabungkan dalam 
laporan. PDF yang dimasukan degnan cara ini akan memiliki header dan footer 
seperti pada halaman lainnya. 

\inpdf{assets/pdfs/include}

Cara lain untuk memasukan PDF adalah dengan menggunakan perintah \bslash putpdf 
dengan satu argumen yang berisi nama berkas pdf. Berbeda dengan perintah 
sebelumnya, PDF yang dimasukan dengan cara ini tidak akan memiliki footer atau 
header seperti pada halaman lainnya. 

\putpdf{assets/pdfs/include}


%-----------------------------------------------------------------------------%
\section{Membuat Perintah Baru}
%-----------------------------------------------------------------------------%
Ada dua perintah yang dapat digunakan untuk membuat perintah baru, yaitu: 
\begin{itemize}
	\item \bslash Var \\
		Digunakan untuk membuat perintah baru, namun setiap kata yang diberikan
		akan diproses dahulu menjadi huruf kapital. 
		Contoh jika perintahnya adalah \bslash Var\{adalah\} makan ketika 
		perintah \bslash Var dipanggil, yang akan muncul adalah ADALAH. 
	\item \bslash var \\
		Digunakan untuk membuat perintah atau baru. 
\end{itemize}

